\documentclass{article}
\usepackage[utf8]{inputenc}
\usepackage{hyperref}
\usepackage[T1]{fontenc}
\usepackage{lmodern}
\usepackage{cmap}
\usepackage[utf8]{inputenc}
\usepackage[english]{babel}
\usepackage{graphicx}
\usepackage{caption}
\usepackage{subcaption}

\title{Bay Area Bike Share Data Analysis}
\author{Maxim Kovalev\\\href{mailto:maxim.kovalev@2007.auditory.ru?subject=Bay Area BikeShare Data Analysis}{maxim.kovalev@2007.auditory.ru}}
\date{May 2015}


\newcommand{\fix}{\marginpar{FIX}}
\newcommand{\new}{\marginpar{NEW}}


\begin{document}

\maketitle

\begin{abstract}
Here be abstract.
\end{abstract}

\section{Tech specs}
Data for this report was obtained from Bay Area BikeShare Open Data Challenge (although this is not a competition submission), and as of May 11, 2015 could be obtained here: \url{http://www.bayareabikeshare.com/datachallenge-2014}.

Code for all the analytics is published under GPLv3, and can be found here: \url{https://github.com/maxikov/bikedatan}. This repository contains the entire distribution needed to run the code, except for the data itself. To run the code, extract data from ``August 2013 - February 2014'' archive from BikeShare to ``data/02/'' subfolder, and ``March 2014 - August 2014'' to ``data/08''.

This code runs on Python 2.x interpreters, 2.7 or older, but not 3.x. In addition to the standard library, it uses numpy, matplotlib, and mpl\_toolkits.basemap.

\subsection{Zip code data set}

In order to work with the data about users' home zip codes, I downloaded an extra data set of coordinates of US zip codes from \url{https://www.gaslampmedia.com/download-zip-code-latitude-longitude-city-state-county-csv/}. This data set isn't fully complete, which I partially fixed by manually adding some of the commonly occurring in the main data set zip codes, but for future work a more complete set may be benefitial.

\section{Scope}

In this report I primarily focus on the data that can be derived by incorporating the information about users' home coordinates, approximately derived from the zip code. As far as I can tell, none of the Open Data Challenge winners has done that. On contrary, \cite{mousebird} and \cite{planetbabs} have created beautiful and informative tools for studying the graph of rides, so I decided against replicating those already achieved results.

\section{Data set overview}

Bay Area BikeShare (BABS) provides 4 data sets collected over 6 months of their operation (with an addition of the identically structured data sets for 6 more months):
\begin{enumerate}
	\item
		Trip data -- for every ride done on BABS bikes, they provide the time this ride was made, ID of the departure station, and ID of the arrival station. In addition, for those rides made by annual subscribers, subscriber's home zip code is provided.
	\item
		Station data -- for every station, referred to by its ID, latitude and longitude is provided, along with the time the station was put into operation.
	\item
		Weather data -- for every day, various meteorological parameters are provided.
	\item
		Rebalancing data -- for every station, once a minute an observation is made how many bikes it has, and how many available docks it has.
\end{enumerate}

\section{Trip vectors and times}
Figure \ref{fig:travelvector} reveals that relative to the station of departure, riders have a strong tendency to travel from East to West and back more than from North to South. Geographically, this pattern would be expected from San Fransisco, and \cite{babs} confirms that SF accounts for 90\% of all the rides, thus dominating the data set.

As \cite{babs} has also shown, and I confirmed in Figure \ref{fig:sns}, annual subscribers dominate the data set, and exhibit a vastly different pattern compared to non-subscribers. Namely, the pattern of subscriber activity matches the scheme ``morning commute -> going out for lunch -> evening commute'', whereas non-subscribers ride far less, and smoothly peak in the mid-day, matching the pattern of tourist or leisure activity.

\section{Home address distribution}

For annual subscribers, home zip codes are provided. According to \cite{babs}, 80\% of rides are done by the subscribers, which warrants studying them as highly representative of the entire user base. Figure \ref{fig:homezips} shows where the subscribers live -- unsurprisingly, they mostly live in the Bay Area, and are particularly concentrated in the Peninsula, although statistically significant portion of them also lives elsewhere in the Northern California, and some outliers can be found in the entire country.

\section{Distance between home and bike stations}

Figure \ref{fig:startendhome} show the distribution of distances from user's home to the start point of a ride and the end point. By incorporating a 50 km threshold, I make sure to only count Bay Area residents. The distribution is noticeably multimodal, with the largest peak near zero, and the smaller but still significant peak around 20 kilometers. Incidentally, this is the distance between centers of San Francisco and Oakland, which could be a coincidence, but given the 90\% dominance of within-SF rides, it could well indicate that a significant number of people commute from East Bay by BART, and then take a bike.

The distributions of distances from home to end and to start of the travel follow each other very closely. This is not surprising, given that, according to Figure \ref{fig:travelvector}, travels tend to be short, and both points are likely to be close to home. Furthermore, the accuracy of positioning by zip code may be not high enough to really distinguish such small differences -- it may well be the case that both the start and the end points are within the same zip code.

However, a small difference may be noticed near the left border of the distribution. Namely, more rides apparently end start right next to the home than they end, and more rides end slightly faraway from home than they begin. Specifically, there are 57515 rides that start near home, and 55890 ones that end near home.

\begin{figure}
	\centering
	\includegraphics[width=\textwidth]{../travel_vector.png}
	\caption{Distribution of travel vectors, in kilometers}
	\label{fig:travelvector}
\end{figure}

\begin{figure}
	\centering
	\includegraphics[width=\textwidth]{../time_of_day_sub_not_sub.png}
	\caption{Distribution of travel vectors, in kilometers}
	\label{fig:sns}
\end{figure}

\begin{figure}
        \centering
        \begin{subfigure}[b]{0.5\textwidth}
                \includegraphics[width=\textwidth]{../home_zips_world.png}
                \caption{On the world map}
                \label{fig:homezips:world}
        \end{subfigure}%
        ~ %add desired spacing between images, e. g. ~, \quad, \qquad, \hfill etc.
          %(or a blank line to force the subfigure onto a new line)
        \begin{subfigure}[b]{0.5\textwidth}
                \includegraphics[width=\textwidth]{../home_zips_usa.png}
                \caption{On contiguous states map}
                \label{fig:homezips:usa}
        \end{subfigure}
        ~ %add desired spacing between images, e. g. ~, \quad, \qquad, \hfill etc.
          %(or a blank line to force the subfigure onto a new line)
        \begin{subfigure}[b]{0.4\textwidth}
                \includegraphics[width=\textwidth]{../home_zips_norcal.png}
                \caption{On Northern California map}
                \label{fig:homezips:norcal}
        \end{subfigure}
        \begin{subfigure}[b]{0.4\textwidth}
                \includegraphics[width=\textwidth]{../home_zips_bay_area.png}
                \caption{On Bay Area map}
                \label{fig:homezips:}
        \end{subfigure}
        \caption{Coordinates of home zip codes of subscribers}
        \label{fig:homezips}
\end{figure}

\begin{figure}
	\centering
	\includegraphics[width=\textwidth]{../start_end_home.png}
	\caption{Distribution of distances from user's home to the start point of the trip (blue) and the end point (green)}
	\label{fig:startendhome}
\end{figure}

\begin{figure}
	\centering
	\includegraphics[width=0.8\textwidth]{../rides_from_home_to_home.png}
	\caption{Number of rides per time of the day. Blue are the rides where the user's home is closer to the start point of the ride, and green are the opposite}
	\label{fig:ridestofromhome}
\end{figure}

\begin{figure}
	\centering
	\includegraphics[width=0.8\textwidth]{../empty_full_stations.png}
	\caption{Observations of empty and full stations. Observations are made every minute, so for each bar the height is proportional to the sum of the portions of time each station has stayed empty or full. }
	\label{fig:emptyfull}
\end{figure}


\bibliographystyle{apa}

\begin{thebibliography}{9}

\bibitem{mousebird}
\url{http://mousebirdconsulting.blogspot.ru/2014/04/bay-area-bike-share-data-challenge.html}

\bibitem{planetbabs}
\url{http://www.bayareabikeshare.com/assets/pdf/Bjorn.pdf}

\bibitem{babs}
\url{http://thfield.github.io/babs/}


\end{thebibliography}




\end{document}

